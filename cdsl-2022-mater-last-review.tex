\documentclass[twocolumn, a4paper, 9pt]{jarticle}
%
\usepackage[dvips]{graphicx}
\usepackage{latexsym}

\makeatletter
\def\section{\@startsection{section}{1}{\z@}{2ex plus .2ex minus .2ex}%
  {.5ex plus .2ex minus .2ex}{\large\bfseries}}
\def\thesection{\arabic{section}.}
\def\subsection{\@startsection{subsection}{1}{\z@}{.7ex plus .2ex minus .2ex}%
  {.5ex plus .2ex minus .2ex}{\normalsize\bfseries}}
\def\thesubsection{\arabic{section}.\arabic{subsection}}
\def\thefootnote{\fnsymbol{footnote}}
\makeatother

% ipsj-kansai
\setlength{\topmargin}{-10mm} % 15mm - 1in
\setlength{\headheight}{5mm}
\setlength{\headsep}{5mm}
\setlength{\oddsidemargin}{-7mm} % 18mm - 1in
\setlength{\evensidemargin}{-7mm} % 18mm - 1in
\setlength{\textheight}{247mm} % 297 - 25(top) - 25(bottom)
\setlength{\textwidth}{174mm} % 210 - 18*2
\setlength{\columnsep}{7mm}

\usepackage{fancyhdr}
%\pagestyle{empty}
\pagestyle{fancy}
\thispagestyle{fancy}
\lhead[]{\fontsize{9pt}{9pt} \sf クラウド・分散システム研究室}
\rhead[]{\fontsize{9pt}{9pt} 2022年度修士論文 最終審査会講演原稿}
\cfoot[]{\fontsize{9pt}{9pt} \copyright Cloud and Distributed Systems Laboratory 2023}
\renewcommand{\headrulewidth}{0pt}

\begin{document}
\twocolumn[
  \begin{center}
    {
      {\bf \LARGE (最終審査会のタイトルにする)\\} % 14.4pt
      %      \vspace{0.5ex}
      %      {\bf \sf \large Request of the Manuscript for IPSJ Kansai-Branch Convention 2014 \\}
    }
    \vspace{1.5ex}
    {
      \fontsize{10.5pt}{10pt}
      \begin{tabular}{c}
        情報 太郎(G2121NNN)$^\dag$
      \end{tabular}
      %      \and
      %      \begin{tabular}{c}
      %        串田 高幸$^\dag$
      %      \end{tabular}
    }
    \vspace{1.5ex}
  \end{center}
]

\insert\footins{
  \noindent
  \small % 9pt
  \begin{tabular}{r p{23em}}
    $^\dag$ & 東京工科大学大学院バイオ・情報メディア研究科コンピュータサイエンス専攻
  \end{tabular}
}

% 1
\section*{はじめに}
背景や前提を記述する.

上記の「講演原稿タイトル」は,自分のタイトルに変更してください.
「情報 太郎(G2121NNN)」は,自分の名前と学生番号を入れてください.
メールアドレス「g2121nnn@edu.teu.ac.jp」は,入れてなくてOKです.

これは,中間発表会・中間審査会の講演原稿のテンプレートです.
発表資料は,A4の2ページです.図や記述を2ページに収めるための
構成になっています.これらの章以外が必要であれば,
それぞれの章のなかにサブセクションとして追加してください.

% 2
\section*{課題}
解決すべき技術的な課題が,何であるかを明確に記述する.

% 3
\section*{提案方式}
前記の課題を解決するための提案や方式やアルゴリズムと,
その優位性について記述する.図や表を入れて,丁寧に説明する.

% 4
\section*{ソフトウェアの設計と実装}
ソフトウェアを,どのように実装したかについて具体的に
図を入れて,丁寧に明確に説明する.
また,実装したソフトウェアのコンポーネントの図を入れる.

\section*{評価方法と結果}
どのような実験をしたか,評価方法について具体的に説明する.
この評価によって,どのように検証したかについても具体的に説明する.
また,実験結果を分析した結果についてグラフを使って記述する.

% 5
\section*{議論}
提案方式において,まだ必要であることやその方法についての議論を記述する.

% 6
\section*{おわりに}
課題,提案方式,結果を簡潔にまとめて,現在までの貢献が,何かを明確に記述する.



%本文中の参考文献は,文の最後のピリオドの前におく\cite{okumura}\cite{companion}.
%%
%% ここから参考文献.
%%
\begin{thebibliography}{10}
  {\small
  \bibitem{okumura}
  奥村晴彦:改訂第5版\LaTeXe 美文書作成入門,
  技術評論社(2010).
  \bibitem{companion}
  Goossens, M., Mittelbach, F. and Samarin, A.:
  {\it The LaTeX Companion},
  Addison Wesley, Reading, Massachusetts (1993).
  }
\end{thebibliography}

\end{document}
